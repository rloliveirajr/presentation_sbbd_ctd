\documentclass{vldb}
\usepackage{balance}

\usepackage{pst-3dplot}

\usepackage[english]{babel}
%\usepackage{epsfig}
\usepackage{multirow}
\usepackage{amsmath}
\usepackage{amssymb}
\usepackage{amsfonts}
\usepackage{stmaryrd}
%\usepackage{graphics}
%\usepackage{graphicx}
%\usepackage{float}
\usepackage{url}
\usepackage{cite}
%\usepackage{units}
\usepackage{subfigure}
\usepackage{algorithm}
\usepackage{algorithmicx}
\usepackage[noend]{algpseudocode}
\usepackage{epstopdf}

\subfigcapmargin=9pt

\newcounter{lineindex}

\begin{document}

\title{Economically-Efficient Sentiment Stream Analysis}

\numberofauthors{2}
\author{
\alignauthor
Roberto Louren\c{c}o, Adriano Veloso\\Wagner Meira Jr., Adriano Pereira\\Renato Ferreira\\
       \affaddr{Computer Science Department}\\
       \affaddr{Universidade Federal de Minas Gerais}\\
       \email{\{robertolojr, adrianov, meira,\\adrianoc, renato\}@dcc.ufmg.br}
\alignauthor
Srinivasan Parthasarathy\\
       \affaddr{Dept. of Computer Science and Engineering and Department of Biomedical Informatics}\\
       \affaddr{The Ohio-State University}\\
       \email{srini@cse.ohio-state.edu}
}

\date{}

\maketitle

\begin{abstract}
Text-based social media channels, such as Twitter, produce
torrents of opinionated data about the most diverse topics and entities.
The analysis of such opinionated data (aka. sentiment analysis) is quickly becoming an essential feature in recommendation systems
and search engines.
A prominent approach to sentiment analysis is based on the application of classification techniques, that is,
content is classified according to
the implicit attitude of the writer with respect to a query term.
A major challenge, however, is that Twitter (and other social channels)
follows the data stream model, and thus the classifier must operate
with limited resources, including labeled data and time for building classification models. Also challenging is the fact that
sentiment distribution may change as the stream evolves.
In this paper we address these challenges by proposing selective sampling algorithms that separate relevant training instances at each time step, so that the corresponding training window is kept small while providing to the classifier the capabilities to suit itself to sentiment drifts (i.e., adaptiveness), and also to recover itself from drifts (i.e., memorability). Simultaneously providing both capabilities to the classifier is a hard task, since this may
lead to a conflicting-objective problem, in which the attempt to improve memorability further may result in worsening adaptiveness (or vice-versa). Our proposed selective sampling approaches employ basic notions of economic efficiency, such as {\em Pareto} and {\em Kaldor-Hicks} efficiency criteria, in order to find a proper balance between adaptiveness and memorability. We performed the analysis of major events in recent years that reverberated on Twitter,
and the comparison against the state-of-the-art reveals improvements in terms of error reduction (up to 14\%) and reduction of training resources (by two orders of magnitude).
\end{abstract}

%\category{H.3.1}{Information Storage and Retrieval}{Content Analysis}
%\category{I.5.2}{Pattern Recognition}{Classifier Design and Evaluation}

%\terms{Algorithms, Experimentation, Measurement, Performance}

%\keywords{Sentiment Analysis, Economic Efficiency}

\section{Introduction}

The need for real-time text analytics is clear and present given the ubiquitous reach of social media sites like Facebook and Twitter. Specifically, recognizing customer sentiment in real-time and enabling advertising on-the-fly have the potential to be a breakthrough technology~\cite{forbes}.
%(\url{http://www.forbes.com/sites/lorikozlowski/2012/04/24/twitter-and-our-feelings-real-time-sentiment-analysis}).
Early examples of such technology in use were demonstrated in this year's National Football League's Superbowl (a premier sporting event in the USA) where a well known manufacturer of {\em Oreo} cookies took advantage of a third quarter blackout (and associated Twitter sentiment) to embed a contextual advertisement. Another example at the same event was the advertisement for a Hollywood movie, where, based on the initial advertisement which happened before the start of the first quarter (and associated Twitter sentiment), the decision on which of several possible advertisements to run later on in the program was apparently taken as a runtime decision.
Examples like these are likely to occur more frequently due to
%Examples such as these are likely to occur more frequently in the near future, going beyond advertising.
lightweight and easy communication mechanisms, such as Twitter microblogging, which makes people eager not only to exchange information, but also to convey their opinions and emotions. People watch events together on television, while tweeting out about things happening around them. As a result, opinionated content is created almost at the same time the event is happening in the real world, and becomes available shortly after.  The analysis of such content (aka. sentiment analysis) in order to exploit the aggregate sentiment of the online crowd goes beyond advertising, and is becoming crucial to recommender systems and search engines.

There is a growing trend in performing sentiment analysis using classification-related techniques: a process that automatically
builds a classification model by learning, from a set of previously
labeled data (i.e., the training-set), the underlying characteristics that distinguish one sentiment from another (i.e., happiness, madness, surprise, suspicion). The success of these classifiers
rests on their ability to judge attitude by means of textual-patterns present in the data, which usually appear in the form
of (idiomatic) expressions and combinations of words.
Sentiment analysis over Twitter real-time messages, however, is particularly challenging, because: (i) Twitter follows the data stream model\footnote{
There are three main source streams in Twitter. The Firehose provides all status updates from everyone in real-time.
Spritzer and Gardenhose are two sub-samples of the Firehose. The current sampling rates are 5\% and 15\%, respectively.},
requiring classifiers to operate with limited computing and training resources, 
%(ii) the training-set is potentially noisy, since training messages may be (incorrectly) labeled using either author-provided sentiment indicators (i.e., emoticons and hash tags~\cite{sigir,tags}) or via crowdsourcing annotation,
and (ii)
either sentiment distribution or the characteristics related to certain sentiments may change over time in almost unforeseen ways (i.e., sentiment drift).

%\paragraph*{\bf{Our Approach to Sentiment Stream Analysis}}
\subsection*{Our Approach to Sentiment Stream Analysis}
A possible strategy to cope with the aforementioned challenges is to employ selective sampling algorithms in order to
focus only on the most relevant training examples/messages at
each time step and to creating training sets from which classifiers are built. Such
training sets are kept as small as possible to ensure fast learning times, since a new classifier must be built at each time step, after a new target message arrives.
Also, messages should be
selected so that the resulting training set
provides sufficient resources to enable the resulting classifier to be effective under the occurrence of drifts.
In order to provide sufficient training resources while keeping sets small, our algorithms select training messages by taking into account two
important properties, that we define as adaptiveness and memorability.
Informally, adaptiveness enables the classifier to adapt
itself to drifts, and thus, improving adaptiveness involves incorporating fresh messages into
the current training set, while discarding obsolete ones. Memorability, on the other hand, involves retaining messages belonging to pre-drift distributions,
therefore enabling the classifier to recover itself from drifts.

We hypothesize that adaptiveness and memorability are both necessary to make classifiers robust to drifts. 
However, given their antagonistic natures, improving both properties may lead to a conflicting-objective problem, in which the attempt to improve memorability further may result in worsening adaptiveness.
Thus, we tackle the problem
by proposing selective sampling algorithms based on multi-objective optimization, that is, we propose to select training messages so that the resulting classifier achieves a proper balance between memorability and adaptiveness.
Our algorithms are based on
central concepts in Economics, namely {\em Pareto} and {\em Kaldor-Hicks} efficiency criteria~\cite{palda@book,kaldor,hicks}. The Pareto Efficiency criterion informally states
that ``when some action could be done to make someone better off without
hurting anyone else, then it should be done.'' This action is called Pareto improvement,
and a system is said to be Pareto-Efficient if no such improvement is possible.
The Kaldor-Hicks criterion is less stringent and states that ``when some action could be done to make someone better off, and this could compensate those that are made worse off, then it should be done.''

\subsection*{Contributions and Findings}
%\paragraph*{\bf{Contributions and Findings}}
The main contribution of this paper is to exploit the intuition behind the aforementioned concepts for devising new algorithms for sentiment stream analysis.
In practice, we claim the following benefits and contributions:


\begin{itemize}
\item We formulate simple-to-compute yet effective utility measures that capture the notions of adaptiveness and memorability. For instance, the similarity between messages that are candidate to compose the current training set and the target message, as well as the freshness of the candidate messages, are measures that tend to privilege adaptiveness. In contrast, candidate messages are also randomly shuffled, thus privileging memorability. These utility measures result in a utility space, and the extent to which each candidate message contributes to adaptiveness and memorability depends on where it is placed in this space.
\item We exploit the concept of Pareto Efficiency by separating messages (viewed as points in the utility space) that are not dominated by any other message. These messages compose the Pareto frontier~\cite{palda@book}, and messages lying in this frontier correspond to cases for which no Pareto improvement is possible. These messages privilege either adaptiveness or memorability, and thus they are selected to compose the current training set from which the classifier is built.
\item We exploit the concept of Kaldor-Hicks Efficiency by selecting an additional set of messages that, although not lying in the Pareto frontier, correspond to a positive trade-off between adaptiveness and memorability. These messages are selected to compose the current training set from which the classifier is built.
\item Our algorithms may operate either on an instance-basis or in batch-mode, by employing classification models based on sentiment rules that are kept incrementally as the stream evolves and training sets are modified.
\end{itemize}

%While every Pareto improvement is a Kaldor-Hicks improvement, most Kaldor-Hicks improvements are not Pareto improvements. That is, the set of Pareto improvements is a proper subset of Kaldor-Hicks improvements, which reflects the greater flexibility and applicability of the Kaldor-Hicks criterion relative to the Pareto criterion.

%We exploit these efficiency concepts as follows. Each possible training message is associated with a point in an n-dimensional scattergram (which we call utility space). In this case, a point is represented as $[a_1, a_2, \ldots, a_n]$, where each coordinate $a_i$ corresponds to a quantity that is either related to adaptiveness or memorability.
%Points that are not dominated by any other point in the scattergram compose the Pareto frontier~\cite{palda@book}. Points lying in the frontier correspond to cases for which no Pareto improvement is possible, being therefore messages likely to privilege either adaptiveness or memorability at a particular time step. Dominated messages are discarded, while messages in the frontier are selected to compose the current training set, from which the classifier is built. 
%A broader frontier, composed of additional training messages, can be built using the Kaldor-Hicks criterion. The process of building the classifier from the messages in the (either Pareto or Kaldor-Hicks) frontier is repeated every time a new target message arrives in the stream.

To evaluate the effectiveness of our algorithms, we
performed experiments using Twitter data collected from three important events in 2010, spanning different sentiments expressed in different languages.
Results show that our algorithms make classifiers extremely effective, with gains in prediction performance that are up to 14\% when compared against the state-of-the-art. Further, the amount of training resources needed is decreased by two orders of magnitude.

\section{Related Work}

In the data stream model, data arrives at high speed and algorithms must
work in real time and with limited resources. Further, in some domains,
algorithms must deal either with burst detection~\cite{shasha} and concept drift (i.e.,
data which nature or distribution change over time). \v{Z}liobait\.{e}
\cite{DBLP:journals/corr/abs-1010-4784} categorizes such drifts as sudden, gradual,
incremental and recurring. When data distribution or nature change over time, its relevance
must be recalculated to avoid harming the model. This kind of data stream is known as
evolving data streams.

Many techniques have been proposed to allow accurate classification in evolving data
streams.  N\'{u}\~{n}ez et al. \cite{Nunez:2007:LEU:1314498.1390328} proposed a
method for keeping a variable training window by adjusting internal
structures of decision trees. An ensemble of Hoeffding trees have been proposed
in \cite{Bifet:2012:ERH:2089094.2089106}, each tree is limited to a small subset of
attributes. Gama et al.\cite{Gama2009a} proposed a mechanism to discard old
information based on sliding windows. Bifet et al.  \cite{Bifet2007,
Bifet:2009:ALE:1617420.1617445} proposed an adaptive sliding window algorithm,
called ADWIN, suitable for data streams with sudden drifts.
The approach presented in
\cite{Koychev00gradualforgetting} suggests that a time-based
forgetting function, which makes more recent observations more significant,
provides adaptiveness to the classifier.
Klinkenberg \cite{Klinkenberg:2004:LDC:1293831.1293836} compares example selection, often
used in windowing approaches with example weights. Experiments
show that both approaches are effective.
In \cite{sigir} the
authors proposed an approach based on a training augmentation procedure, which
automatically incorporates
relevant training messages into the training-set.
%Classification models are produced
%on-the-fly using association rules, which are kept up-to-date in an incremental
%fashion, so that at any given time the model properly reflects the sentiments
%in the event being analyzed.

Some works have focused on feature similarity, such as Torres et al.
\cite{Torres:2011:CMD:2025756.2025758} that studied different methods for data
stream classification and proposed a new way of keeping the representative
data models based on similarity measures.
Feng et al. \cite{Feng2013} extracted the
concept from each data block using feature similarity
probabilities. 
Masud et al. \cite{Masud:2008:PAC:1510528.1511337} proposed a novel technique
to overcome the lack of labeled examples by building models
from unlabeled instances and a small amount of labeled ones.
Zhu et al. \cite{5440901} employed active learning to produce a
classifier ensemble that selects labeled instances from data streams to
build classifiers. Also, in \cite{6414645,Indre2011k} active
learning approaches are presented for data streams that explicitly handle
concept drifts. They are based on uncertainty~\cite{uncertainty}, dynamic allocation 
of labeling efforts over time, and randomization of the search space. \v{Z}liobait\.{e} et
al.  \cite{DBLP:journals/jmlr/ZliobaiteBHP11} proposed a system that implements active
learning strategies, extending the Massive Online Analysis (MOA) framework \cite{moa}.

Works above cited attempt to face concept drift in data stream through manipulation of
classifiers, with mechanisms such as training windows and decay functions,
active learning and sampling. In this paper we present new algorithms that
select high-utility examples in order to provide adaptiveness and memorability
to the classifier. In order to balance adaptiveness
and memorability, we formalized this issue as a multi-objective problem. The sample
selection is performed using economic efficiency criteria: Pareto and Kaldor-Hicks.
%Also, we discuss algorithms that operate under instance-basis and batch-mode scenarios. In particular, the batch-mode scenario enables our algorithms to
%reduce labeling effort.
We did not find in the recent literature
approaches that employ multi-objective models based on economic efficiency criteria to
deal with issues in the data stream environment.

\section{Algorithms}

In this section we present novel selective sampling approaches for learning classifiers to distinguish between different sentiments expressed in Twitter messages. We start by discussing models based on specialized association rules.
Then we present measures for adaptiveness and memorability, and describe the message utility space. Finally, we discuss Pareto and Kaldor-Hicks criteria, and algorithms that select training messages using these criteria.

\subsection{Sentiment Stream Analysis}

In our context, the task of learning sentiment streams is precisely defined as follows. At time step $n$, we have as input
a training set referred to as $\mathcal{D}_n$, which consists of
a set of records of the form $<d,s_i>$, where $d$ is a
message (represented as a list of terms),
and $s_i$ is the sentiment implicit in $d$.
The sentiment variable $s$ draws its values from a pre-defined, fixed and discrete set of possibilities (e.g., $s_1$,
$s_2$, $\ldots$, $s_k$).
The training set is used to build a classifier relating textual patterns in the messages to their corresponding sentiments.
A sequence of future messages referred to as $\mathcal{T}=\{t_n, t_{n+1}, \ldots\}$, consists of messages
for which only their terms are known, while the corresponding sentiments are unknown.
The classifier obtained from $\mathcal{D}_n$ is used
to score the sentiments for message $t_n$ in $\mathcal{T}$.
Messages in $\mathcal{T}$ are eventually incorporated into the next training set.

There are countless strategies for devising
a classifier for sentiment analysis. Many of these strategies, however, are not well-suited to deal with data streams. Some are specifically devised for offline classification~\cite{trees,cortes}, and this is problematic because producing classifiers on-the-fly would be unacceptably costly. %Even updating the models in scenarios with high-speed streams would be
%excessively lengthy.
In such circumstances, alternate classification strategies may become more convenient~\cite{calibrated}.

\subsection{Sentiment Rules and Classifiers}

Next we describe classifiers composed of association rules, and how these rules are used for sentiment-scoring. Such classifiers are built on-the-fly~\cite{hipc,lac}, being thus well-suited for sentiment stream analysis, as shown in~\cite{sigir}.

\paragraph*{\bf{Definition 1}}
A sentiment rule is a specialized association rule $\mathcal{X}\xrightarrow{}s_i$, where the antecedent $\mathcal{X}$ is a set of terms (i.e., a termset), and the consequent $s_i$ is the predicted sentiment. The domain for $\mathcal{X}$ is the vocabulary of the training set $\mathcal{D}_n$.
The support of $\mathcal{X}$ is denoted as $\sigma(\mathcal{X})$, and is the number of messages in $\mathcal{D}_n$ having $\mathcal{X}$ as a subset. The confidence of rule $\mathcal{X}\xrightarrow{}s_i$ is denoted as $\theta(\mathcal{X}\xrightarrow{}s_i)$ and is given as $\displaystyle\frac{\sigma(\mathcal{X}\cup s_i)}{\sigma(\mathcal{X})}$.\\

\subsection*{Sentiment Scoring}

We denote as $\mathcal{R}(t_n)$ the classifier obtained at time step $n$, by extracting rules from $\mathcal{D}_n$.
Basically, the classifier is a poll of rules, and each rule $\{\mathcal{X}\xrightarrow{}s_i\}\in\mathcal{R}(t_n)$ is a vote given for sentiment $s_i$. Given message $t_n$, a rule is a valid vote if it is applicable to $t_n$.

\paragraph*{\bf{Definition 2}} A rule $\{\mathcal{X}\xrightarrow{}s_i\}\in\mathcal{R}(t_n)$ is said to be applicable to message $t_n\in\mathcal{T}$ if
all terms in $\mathcal{X}$ are in $t_n$.\\

We denote as $\mathcal{R}_a(t_n)$ the set of rules in $\mathcal{R}(t_n)$ that are applicable to message $t_n$. Thus, only rules in $\mathcal{R}_a(t_n)$ are considered as valid votes when scoring sentiments in $t_n$.
Further, we denote as $\mathcal{R}^{s_i}_a(t_n)$ the subset of $\mathcal{R}(t_n)$ containing only rules predicting sentiment $s_i$.
Votes in $\mathcal{R}^{s_i}_a(t_n)$ have different weights, depending on the confidence of the corresponding rules. The weighted votes for sentiment $s_i$ are averaged, giving the score for $s_i$ with regard to $t_n$:

\begin{equation} \label{eq1}
s(t_n, s_i)=\displaystyle\sum \displaystyle\frac{\theta(\mathcal{X}\xrightarrow{}s_i)}{|\mathcal{R}^{s_i}_a(t_n)|}
\end{equation}

Finally, the scores are normalized, thus giving the likelihood of sentiment $s_i$ being the attitude in message $t_n$:

\begin{equation}
\label{eq:prob}
\hat{p}(s_i|t_n)=\displaystyle\frac{s(t_n, s_i)}{\displaystyle\sum^{k}_{j=1} s(t_n, s_j)}
\end{equation}

\subsection*{Rule Extraction}

The simplest approach to rule extraction is the offline one. In this case, rule extraction is divided into two steps: support counting and confidence computation. Once the support $\sigma(\mathcal{X})$ is known, it is straightforward to compute the confidence $\theta(\mathcal{X}\xrightarrow{}s_i)$ for the corresponding rules~\cite{eclat}.
There are several smart support-counting strategies~\cite{rules,fp,eclat}, and many fast implementations~\cite{fimi} that can be used.
We employ the vertical counting strategy, which is based on the use of inverted lists~\cite{vertical}.
Specifically, an inverted list associated with termset $\mathcal{X}$, is denoted as $\mathcal{L}(\mathcal{X})$, and contains the identifiers of the messages in $\mathcal{D}_n$ having termset $\mathcal{X}$ as a subset. An inverted list $\mathcal{L}(\mathcal{X})$ is obtained by performing the intersection of two proper subsets of termset $\mathcal{X}$. The support of termset $\mathcal{X}$ is given by the cardinality of $\mathcal{L}(\mathcal{X})$, that is, $\sigma(\mathcal{X})=|\mathcal{L}(\mathcal{X})|$.

Usually, the support for different sets of terms in $\mathcal{D}_n$ are computed in a bottom-up way, which
starts by scanning all messages in $\mathcal{D}_n$
and computing the support of each term in isolation.
In the next iteration,
pairs of terms are enumerated, and their support values are calculated by performing the intersection of the corresponding proper subsets.
The search for sets of terms proceeds, and the enumeration process is repeated until the support values for all sets of terms in $\mathcal{D}_n$ are finally computed.
Obviously, the number of rules increases exponentially with the size of the vocabulary (i.e., the number of distinct terms in $\mathcal{D}_n$), and
computational cost restrictions have to be imposed during
rule extraction. Typically, the search space for rules is restricted by pruning
rules that do not appear frequently in $\mathcal{D}_n$ (i.e., the minimum support approach). While such restrictions make rule extraction feasible, they also lead to lossy classifiers, since some rules are pruned and therefore are not included into $\mathcal{R}(t_n)$.

\paragraph*{\bf{Online Rule Extraction}}
An alternative to offline rule extraction is to extract rules on-the-fly. Such alternative, which we call online rule extraction, has several advantages~\cite{sigir}. For instance, it becomes possible to efficiently extract rules from $\mathcal{D}_n$ without performing support-based pruning.
The idea behind online rule extraction is to ensure that only applicable rules are extracted by projecting $\mathcal{D}_n$ on a demand-driven basis. More specifically, rule extraction is delayed until a message $t_n\in\mathcal{T}$ is given. Then, terms in $t_n$ are used as a filter which configures $\mathcal{D}_n$ in a way that only rules that are applicable to $t_n$ can be extracted. This filtering process produces a projected training-set, denoted as $\mathcal{D}^{*}_n$, which contains only terms that are present in message $t_n$.

\paragraph*{\bf{Lemma 1}}
All rules extracted from $\mathcal{D}^{*}_n$ are applicable to $t_n$.

\paragraph*{\bf{Proof}}
Since all training messages in $\mathcal{D}^{*}_n$ contain only terms that are
present in message $t_n$, the existence of a rule
$\mathcal{X}\xrightarrow{}s_i$ extracted from $\mathcal{D}^{*}_n$, such that
$\mathcal{X}\nsubseteq t_n$, is impossible. $\blacksquare$\\

Lemma 1 implies that online rule extraction assures that $\mathcal{R}(t_n)=\mathcal{R}_a(t_n)$.
The next theorem states that
search space for rules induced by $\mathcal{D}^{*}_n$ is much narrower than the search space for rules induced by $\mathcal{D}_n$. Thus, rules
can be efficiently extracted from $\mathcal{D}^{*}_n$, no matter the minimum-support value (which can be arbitrary low).

\paragraph*{\bf{Theorem 1}}
The number of rules extracted from $\mathcal{D}^{*}_n$ increases polynomially with
the number of distinct terms in $\mathcal{D}_n$.

\paragraph*{\bf{Proof}}
Let $k$ be the number of distinct terms in
$\mathcal{D}_n$.
Since an arbitrary message $t_n\in\mathcal{T}$ contains at most $l$
terms (with $l\ll k$), then any rule applicable to $t_n$
can have at most $l$ terms in its antecedent. That is, for any
rule $\{\mathcal{X}\xrightarrow{}s_i\}$, such that $\mathcal{X}\subseteq t_n$,
$|\mathcal{X}|\le l$. Consequently, the number of
possible rules that are applicable to $t_n$ is
$l+{{l}\choose{2}}+\ldots+{{l}\choose{l}}=O(2^l)\ll O(k^l)$.
Thus, the number of applicable rules increases
polynomially in $k$.
$\blacksquare$\\

\paragraph*{\bf{Extending Classifiers Dynamically}}
Let $\mathcal{R}=\{\mathcal{R}(t_1)\cup\mathcal{R}(t_2)$ $\cup\ldots\cup\mathcal{R}(t_n)\}$.
With online rule extraction, 
$\mathcal{R}$ is extended dynamically as messages $t_i\in\mathcal{T}$ are processed. Initially $\mathcal{R}$ is empty; a classifier $\mathcal{R}_{t_i}$ is appended to $\mathcal{R}$ every time a message $t_i$ is processed.
Producing a classifier $\mathcal{R}(t_i)$ involves extracting rules from the corresponding training-set. This operation has a significant computational cost, since it is necessary perform multiple accesses to $\mathcal{D}_i$.
Different messages in $\mathcal{T}=\{t_1, t_2, \ldots, t_m\}$ may demand different classifiers $\{\mathcal{R}_{t_1}, \mathcal{R}_{t_2}, \ldots,$ $\mathcal{R}_{t_m}\}$,
but different classifiers may share some rules (i.e., $\{\mathcal{R}_{t_i}\cap\mathcal{R}_{t_j}\}\ne\emptyset$).
In this case,
memorization is very effective in avoiding work replication,
reducing the number of data access operations.
Thus, before extracting
rule $\mathcal{X}\xrightarrow{}s_i$, the classifier first checks whether this rule is
already in $\mathcal{R}$. If an entry is found, then the rule
in $\mathcal{R}$ is used instead of extracting it from the training-set. If it is not found, the rule is
extracted from the training-set and then it is inserted into $\mathcal{R}$.

%\paragraph*{\bf{Model Maintenance}}
%Rules in $\mathcal{R}$ may become invalid when reliable labeled messages $<t, s_i>$ are included into $\mathcal{D}$. As a result,
%$\mathcal{R}$ has to be updated properly. We propose to maintain the model up-to-date incrementally, so that the updated model is exactly the same one that
%would be obtained by re-constructing it from scratch.
%
%Update speed is a key issue in model maintenance, and a challenge that threatens the efficiency of our approach is that the model may be composed of a potentially large number of rules, and updating all these rules may be unacceptably costly in a streaming environment.
%Fortunately, not all rules in $\mathcal{R}$ have to be updated.
%
%\paragraph*{Lemma 2} The inclusion of a labeled message $<t, s_i>$ into $\mathcal{D}$ does not change the value of $\sigma(\mathcal{X})$, for any termset $\mathcal{X}\not\subset t$.
%
%\paragraph*{Proof} Since $\mathcal{X}\not\subset t$, the number of messages in $\mathcal{D}$ having $\mathcal{X}$ as a subset is essentially the same as in $\{\mathcal{D}\cup t\}$. $\blacksquare$
%
%\paragraph*{Lemma 3} The inclusion of a labeled message $<t, s_i>$ into $\mathcal{D}$ does not change the value of $\theta(\mathcal{X}\xrightarrow{}s)$,
%for any rule $\{\mathcal{X}\xrightarrow{}s\}\in\mathcal{R}$ for which $\mathcal{X}\not\subset t$,
%$\forall s\in\{s_1, s_2,\ldots, s_k\}$.
%
%\paragraph*{Proof} Comes directly from the fact that confidence is invariant under the null-addition operation~\cite{measure1}.
%$\blacksquare$\\
%
%From Lemmas 2 and 3, the number of rules that have to be updated due to the inclusion of labeled message $<t, s_i>$, is bounded by the number of possible termsets in $t$. Since most of the messages that are included into $\mathcal{D}$ contain only a very small fraction of all possible termsets, the inclusion of an arbitrary message $t$ corresponds to a null-addition to most of the rules in $\mathcal{R}$. The following lemma states exactly the rules in $\mathcal{R}$ that have to be updated.
%
%\paragraph*{Lemma 4} The only and all the rules in $\mathcal{R}$ that must be updated due to the inclusion of labeled message $<t, s_i>$ are those in $\mathcal{R}_t$.
%
%\paragraph*{Proof} All rules $\{\mathcal{X}\xrightarrow{}s_i\}\in\mathcal{R}$ that have to be updated due to the inclusion of $<t, s_i>$ are those for which $\mathcal{X}\subseteq t$. By definition, $\mathcal{R}_t$ contains only and all such rules. $\blacksquare$\\
%
%Once rules 
%$\{\mathcal{X}\xrightarrow{}s\}\in\mathcal{R}_t$ are retrieved from $\mathcal{R}$,
%updating the corresponding values for $\sigma(\mathcal{X})$ and $\theta(\mathcal{X}\xrightarrow{}s)$
% is a simple operation. It suffices to iterate on $\mathcal{R}_t$ and increment the values of $\sigma(\mathcal{X})$ and $\sigma(\mathcal{X}\cup s)$. The corresponding values for $\theta(\mathcal{X}\xrightarrow{}s)$
%are obtained by computing $\frac{\sigma(\mathcal{X}\cup s)}{\sigma(\mathcal{X})}$
%$\forall s\in\{s_1, s_2, \ldots, s_k\}$.

\subsection{Utility Space and Selective Sampling}

Our approach to sentiment stream analysis is based on selecting high-utility messages to compose the training set at each time step. Training sets must provide adaptiveness and memorability to the corresponding classifiers. Improving adaptiveness and memorability simultaneously, however, is a conflicting-objective problem. Instead, our approaches create training sets that balance between adaptiveness and memorability. Specifically, at each time step, candidate messages are placed into an n-dimensional space, in which each dimension corresponds to a utility measure which is either related to adaptiveness or memorability.

\subsection*{Utility Measures}
%\paragraph*{\bf{Utility Measures}}
At each time step, the classifier must score sentiments that are expressed in the target message. Some of the utility measures we are going to discuss next are based on the distance to the target message. By minimizing such distance we are essentially maximizing adaptiveness, since the selected messages are similar to the target message. As for memorability, we are going to discuss a utility measure based on randomly shuffling candidate messages:

\begin{itemize}
\item{\bf{Distance in space} $-$}
The similarity between the target message $t_n$ and an arbitrary message $t_j$ is given by the number of rules in the classifier $\mathcal{R}_a(t_n)$ that are also applicable to $t_j$.
Differently from traditional measures such as cosine and Jaccard~\cite{baeza99modern}, the rule-based similarity considers not only isolated terms, but also combination of terms.
Thus, the utility of message $t_j$ is given as:

\begin{equation}
%U_s(t_j)=\displaystyle\frac{|\{r\in\mathcal{R}_a(t_n)\mbox{ such that }r\mbox{ is applicable to }t_j\}|}{|\{\mathcal{R}_a(t_n)\}|}
U_s(t_j)=\displaystyle\frac{|\mathcal{R}_a(t_n)\cap\mathcal{R}_a(t_j)|}{|\{\mathcal{R}_a(t_n)\}|}
\end{equation}

\item{\bf{Distance in time} $-$}
Let $\gamma(t_j)$ be a function that returns the time in which message $t_j$ arrived. The utility of message $t_j$ is given as: 

\begin{equation}
U_t(t_j)=\displaystyle\frac{\gamma(t_j)}{\gamma(t_n)}
\end{equation}

\item{\bf{Memorability} $-$}
In order to provide memorability, the training set must contain messages posted in different time periods. A simple way to force this is to generate a random permutation of the candidate messages, that is, randomly shuffling the candidate messages~\cite{permutation}.
Let $\alpha(t_j)$ be a function that returns the position of message $t_j$ in the shuffle.
The utility of message $t_j$ is given as:

\begin{equation}
U_r(t_j)=\frac{\alpha(t_j)}{|\mathcal{D}_n|}
\end{equation}



\end{itemize}

Each candidate message is judged based on these three utility measures. The need to judge one situation better than another motivates much of Economics, and next we discuss concepts from Economics and how they can be applied to select messages to compose the training set.

\begin{figure}[htp!]
\centering
\begin{pspicture}(-2.905,-3.4)(5.6,6.05)

\psset{Alpha=50,Beta=20}
\pstThreeDSquare[linecolor=lightgray,fillstyle=solid,fillcolor=lightgray!80](-5.4,0,0)(0,2.7,0)(0,0,2.7)
\pstThreeDSquare[linecolor=lightgray,fillstyle=solid,fillcolor=lightgray!80](-2.7,-3.5625,0)(2.7,0,0)(0,0,2.7)
\pstThreeDSquare[linecolor=lightgray,fillstyle=solid,fillcolor=lightgray!80](-2.7,0,-2.1375)(2.7,0,0)(0,2.7,0)

\pstThreeDLine[linestyle=dashed,linecolor=lightgray,linewidth=0.5pt](-2.7,0,2.7)(-5.4,0,2.7)
\pstThreeDLine[linestyle=dashed,linecolor=lightgray,linewidth=0.5pt](-2.7,0,0)(-5.4,0,0)
\pstThreeDLine[linestyle=dashed,linecolor=lightgray,linewidth=0.5pt](-2.7,2.7,0)(-5.4,2.7,0)

\pstThreeDLine[linestyle=dashed,linecolor=lightgray,linewidth=0.5pt](-2.7,0,2.7)(-2.7,-3.5625,2.7)
\pstThreeDLine[linestyle=dashed,linecolor=lightgray,linewidth=0.5pt](0,0,2.7)(0,-3.5625,2.7)
\pstThreeDLine[linestyle=dashed,linecolor=lightgray,linewidth=0.5pt](-2.7,0,0)(-2.7,-3.5625,0)
\pstThreeDLine[linestyle=dashed,linecolor=lightgray,linewidth=0.5pt](0,0,0)(0,-3.5625,0)

\pstThreeDLine[linestyle=dashed,linecolor=lightgray,linewidth=0.5pt](-2.7,0,0)(-2.7,0,-2.025)
\pstThreeDLine[linestyle=dashed,linecolor=lightgray,linewidth=0.5pt](-2.7,2.7,0)(-2.7,2.7,-2.025)
\pstThreeDLine[linestyle=dashed,linecolor=lightgray,linewidth=0.5pt](0,0,0)(0,0,-2.025)
\pstThreeDLine[linestyle=dashed,linecolor=lightgray,linewidth=0.5pt](0,2.7,0)(0,2.7,-2.025)

\pstThreeDLine[linecolor=lightgray,linewidth=0.5pt](-2.7,0,0)(0,0,0)
\pstThreeDLine[linecolor=lightgray,linewidth=0.5pt](-2.7,0,0)(-2.7,2.7,0)
\pstThreeDLine[linecolor=lightgray,linewidth=0.5pt](-2.7,0,0)(-2.7,0,2.7)


\rput[c](0.60,2.45){$\bullet$}
\rput[c](1.60,2.58){$\bullet$}
\rput[c](2.30,2.41){$\bullet$}
\rput[c](3.16,1.55){$\bullet$}
\rput[c](1.82,1.30){$\bullet$}
\rput[c](0.80,1.64){$\bullet$}

\rput[c](0.60,1.45){$\bullet$}
\rput[c](1.10,1.18){$\bullet$}
\rput[c](2.30,0.41){$\bullet$}
\rput[c](3.05,0.25){$\bullet$}
\rput[c](2.82,0.35){$\bullet$}
\rput[c](1.82,0.40){$\bullet$}
\rput[c](1.82,0.10){$\bullet$}
\rput[c](1.72,-0.10){$\bullet$}
\rput[c](0.90,0.74){$\bullet$}
\rput[c](0.80,0.64){$\bullet$}

%\rput[c](1.6,4.1){Pareto frontier}
%\pscurve[arrowscale=2,linecolor=gray]{->}(2.11,1.51)(4,1.4)(1.6,4)

%\pspolygon[linestyle=none, linewidth=0pt,fillstyle=solid,fillcolor=black!50](0.60,2.45)(1.60,2.58)(2.30,2.41)(3.16,1.55)(1.82,1.30)(0.80,1.64)

\pstThreeDPut(-2.7,0,0){
\pstThreeDLine[linecolor=lightgray,linewidth=0.5pt](0,0,2.7)(2.7,0,2.7)
\pstThreeDLine[linecolor=lightgray,linewidth=0.5pt](0,0,0)(0,0,2.7)
\pstThreeDLine[linecolor=lightgray,linewidth=0.5pt](0,0,0)(2.7,0,0)
\pstThreeDLine[linecolor=lightgray,linewidth=0.5pt](0,0,0)(2.7,0,0)
\pstThreeDLine[linecolor=lightgray,linewidth=0.5pt](0,0,0)(0,2.7,0)
\pstThreeDLine[linecolor=lightgray,linewidth=0.5pt](0,2.7,2.7)(0,2.7,0)
\pstThreeDLine[linecolor=lightgray,linewidth=0.5pt](2.7,0,2.7)(2.7,2.7,2.7)
\pstThreeDLine[linecolor=lightgray,linewidth=0.5pt](0,2.7,2.7)(2.7,2.7,2.7)
\pstThreeDLine[linecolor=lightgray,linewidth=0.5pt](0,0,2.7)(0,2.7,2.7)
\pstThreeDLine[linecolor=lightgray,linewidth=0.5pt](2.7,0,2.7)(2.7,0,0)
\pstThreeDLine[linecolor=lightgray,linewidth=0.5pt](2.7,2.7,0)(2.7,0,0)
\pstThreeDLine[linecolor=lightgray,linewidth=0.5pt](2.7,2.7,0)(0,2.7,0)
\pstThreeDLine[linecolor=lightgray,linewidth=0.5pt](2.7,2.7,0)(2.7,2.7,2.7)
}

\pstPlanePut[plane=yz](-6,1.5,1.875){$\bullet$}
\pstPlanePut[plane=yz](-4,2,1.875){$\bullet$}
\pstPlanePut[plane=yz](-4,2,2.875){$\bullet$}
\pstPlanePut[plane=yz](-3.5,2,3.2){$\bullet$}
\pstPlanePut[plane=yz](-5.5,1,2.2){$\bullet$}
\pstPlanePut[plane=yz](-5,2.9,2.2){$\bullet$}
\pstPlanePut[plane=yz](-4.5,2.1,1.7){$\bullet$}
\pstPlanePut[plane=yz](-4.5,1.4,1.9){$\bullet$}
\pstPlanePut[plane=yz](-4.7,0.8,1){$\bullet$}
\pstPlanePut[plane=yz](-5.6,0.8,1){$\bullet$}
\pstPlanePut[plane=yz](-5.6,0.8,1.6){$\bullet$}
\pstPlanePut[plane=yz](-5.6,0.8,0.2){$\bullet$}
\pstPlanePut[plane=yz](-5.6,1.8,0.2){$\bullet$}

\pstThreeDPut(-0.75,1.5,-2.25){\pstPlanePut[planecorr=xyrot,plane=xy,RotZ=90](0,0,0){$\bullet$}}
\pstThreeDPut(-0.5,0.5,-2.25){\pstPlanePut[planecorr=xyrot,plane=xy,RotZ=90](0,0,0){$\bullet$}}
\pstThreeDPut(-0.5,1.5,-2){\pstPlanePut[planecorr=xyrot,plane=xy,RotZ=90](0,0,0){$\bullet$}}
\pstThreeDPut(-0.5,2.5,-2){\pstPlanePut[planecorr=xyrot,plane=xy,RotZ=90](0,0,0){$\bullet$}}
\pstThreeDPut(-1.5,2.2,-2){\pstPlanePut[planecorr=xyrot,plane=xy,RotZ=90](0,0,0){$\bullet$}}
\pstThreeDPut(-1,2.5,-1.7){\pstPlanePut[planecorr=xyrot,plane=xy,RotZ=90](0,0,0){$\bullet$}}
\pstThreeDPut(-1.1,1.5,-1.7){\pstPlanePut[planecorr=xyrot,plane=xy,RotZ=90](0,0,0){$\bullet$}}
\pstThreeDPut(-1.1,1.1,-1.9){\pstPlanePut[planecorr=xyrot,plane=xy,RotZ=90](0,0,0){$\bullet$}}
\pstThreeDPut(-1.8,0.4,-2.5){\pstPlanePut[planecorr=xyrot,plane=xy,RotZ=90](0,0,0){$\bullet$}}

\pstThreeDPut(-0.325,-3.25,1.875){\pstPlanePut[planecorr=xyrot,plane=xz,RotZ=180](0,0,0){$\bullet$}}
\pstThreeDPut(-0.5,-3.8,1.5){\pstPlanePut[planecorr=xyrot,plane=xz,RotZ=180](0,0,0){$\bullet$}}
\pstThreeDPut(-1.5,-3.8,1.5){\pstPlanePut[planecorr=xyrot,plane=xz,RotZ=180](0,0,0){$\bullet$}}
\pstThreeDPut(-2.5,-4,1.5){\pstPlanePut[planecorr=xyrot,plane=xz,RotZ=180](0,0,0){$\bullet$}}
\pstThreeDPut(-2.5,-3.5,1.7){\pstPlanePut[planecorr=xyrot,plane=xz,RotZ=180](0,0,0){$\bullet$}}
\pstThreeDPut(-2.5,-3.8,1.1){\pstPlanePut[planecorr=xyrot,plane=xz,RotZ=180](0,0,0){$\bullet$}}
\pstThreeDPut(-0.4,-3,1.5){\pstPlanePut[planecorr=xyrot,plane=xz,RotZ=180](0,0,0){$\bullet$}}
\pstThreeDPut(-2,-3.8,0.7){\pstPlanePut[planecorr=xyrot,plane=xz,RotZ=180](0,0,0){$\bullet$}}
\pstThreeDPut(-2.3,-3.8,0.7){\pstPlanePut[planecorr=xyrot,plane=xz,RotZ=180](0,0,0){$\bullet$}}
\pstThreeDPut(-1.3,-3.8,0.7){\pstPlanePut[planecorr=xyrot,plane=xz,RotZ=180](0,0,0){$\bullet$}}
\pstThreeDPut(-0.3,-3.8,0.7){\pstPlanePut[planecorr=xyrot,plane=xz,RotZ=180](0,0,0){$\bullet$}}
\pstThreeDPut(-0.3,-3.2,2.2){\pstPlanePut[planecorr=xyrot,plane=xz,RotZ=180](0,0,0){$\bullet$}}

\pstThreeDLine{->}(-5.4,0,0)(-5.4,3.0375,0)
\pstThreeDLine{->}(-5.4,0,0)(-5.4,0,3.0375)

\pstPlanePut[plane=yz](-3.2,2.2,0.6){Adaptiveness}
\pstPlanePut[plane=yz](-2.2,2.2,0.6){(Distance in time)}

\pstPlanePut[plane=yz](-6.4,-1,3.1725){(Distance in space)}
\pstPlanePut[plane=yz](-5.4,0,3.1725){Adaptiveness}
\pstThreeDLine{->}(0,0,-2.1375)(-3.1725,0,-2.1375)
\pstThreeDLine{->}(0,0,-2.1375)(0,3.1725,-2.1375)
\pstThreeDPut(-3.1725,0,-2.025){\pstPlanePut[planecorr=xyrot,plane=xy,RotZ=90](0,0,0){Memorability}}

\pstThreeDPut(0.4,-0.1,-2.4){\pstPlanePut[planecorr=xyrot,plane=xy,RotZ=90](0,0,0){Adaptiveness}}
\pstThreeDPut(1.4,0.1,-2.4){\pstPlanePut[planecorr=xyrot,plane=xy,RotZ=90](0,0,0){(Distance in time)}}

\pstThreeDLine{->}(0,-3.5625,0)(-3.1725,-3.5625,0)
\pstThreeDLine{->}(0,-3.5625,0)(0,-3.5625,3.1725)
\pstThreeDPut(0,-2.69,0){\pstPlanePut[planecorr=xyrot,plane=xz,RotZ=180](0,0,0){Memorability}}

\pstThreeDPut(-1,-4.6,3.375){\pstPlanePut[planecorr=xyrot,plane=xz,RotZ=180](0,0,0){(Distance in space)}}
\pstThreeDPut(-0.5,-3.6,3.375){\pstPlanePut[planecorr=xyrot,plane=xz,RotZ=180](0,0,0){Adaptiveness}}
\end{pspicture}
\caption{Illustrative example. The 3D utility space.}
\label{fig:ex1}
\end{figure}


\subsection{Economic Efficiency}

When the society is economically efficient, any changes made to assist one person would harm another. The same intuition could be exploited for the sake of selecting messages to compose the training set at each time step. In this case, a training set is economically efficient if it is only possible to improve memorability at the cost of adaptiveness, and vice-versa~\cite{recsys12,icmr14}.

There is an alternative, less stringent notion of efficiency, which is based on the principle of compensation~\cite{compensation}.
Under new arrangements in the society, some may be better off while others may be worse off.
Compensation holds if those made better off
under the new set of conditions could compensate those made worse off. Next we discuss algorithms that exploit these two notions of economic efficiency in order to select messages to compose the training sets.

\subsection*{Pareto Frontier}
%\paragraph*{\bf{Pareto Frontier}}
Messages that are candidate to compose the training set at time step $n$ are placed in a 3-dimensional space, according to their utility measures, as shown in Figure~\ref{fig:ex1}.
Thus, each message $a$ is a point in such utility space, and is given as $<U_s(a),U_t(a),U_r(a)>$.

\paragraph*{\bf{Definition 3}} Message $a$ is said to dominate message $b$ iff both of the following conditions are hold:
\begin{itemize}
\item $U_s(a)\ge U_s(b)$ and $U_t(a)\ge U_t(b)$ and $U_r(a)\ge U_r(b)$
\item $U_s(a) > U_s(b)$ or $U_t(a) > U_t(b)$ or $U_r(a) > U_r(b)$
\end{itemize}
Therefore, the dominance operator relates two messages so that the result of the
operation has two possibilities as shown in Figure~\ref{fig:ex2} (Left): (i) one message dominates another or (ii) the two
messages do not dominate each other.

\begin{figure*}[htp!]
\centering

\begin{pspicture}(14,4.4)

\psset{Alpha=50,Beta=20}

\rput[c](0.60,2.65){$a$}
\rput[c](0.60,2.45){$\circ$}
\psline[linestyle=dotted]{-}(0.60,2.45)(1.60,2.58)
\rput[c](1.60,2.83){$b$}
\rput[c](1.60,2.58){$\circ$}
\rput[c](2.30,2.41){$\circ$}
\rput[c](3.16,1.55){$\circ$}
\rput[c](1.82,1.30){$\circ$}
\rput[c](0.80,1.64){$\circ$}

\rput[c](0.60,1.45){$\circ$}
\psline[linestyle=dotted]{<-}(1.10,1.28)(1.60,2.58)
\rput[c](1.10,1.00){$c$}
\rput[c](1.10,1.18){$\circ$}
\rput[c](2.30,0.41){$\circ$}
\rput[c](3.05,0.25){$\circ$}
\rput[c](2.82,0.35){$\circ$}
\rput[c](1.82,0.40){$\circ$}
\rput[c](1.82,0.10){$\circ$}
\rput[c](1.72,-0.10){$\circ$}
\rput[c](0.90,0.74){$\circ$}
\rput[c](0.80,0.64){$\circ$}

\pstThreeDPut(-2.7,0,0){
\pstThreeDLine[linecolor=lightgray,linewidth=0.5pt](0,0,2.7)(2.7,0,2.7)
\pstThreeDLine[linecolor=lightgray,linewidth=0.5pt](0,0,0)(0,0,2.7)
\pstThreeDLine[linecolor=lightgray,linewidth=0.5pt](0,0,0)(2.7,0,0)
\pstThreeDLine[linecolor=lightgray,linewidth=0.5pt](0,0,0)(2.7,0,0)
\pstThreeDLine[linecolor=lightgray,linewidth=0.5pt](0,0,0)(0,2.7,0)
\pstThreeDLine[linecolor=lightgray,linewidth=0.5pt](0,2.7,2.7)(0,2.7,0)
\pstThreeDLine[linecolor=lightgray,linewidth=0.5pt](2.7,0,2.7)(2.7,2.7,2.7)
\pstThreeDLine[linecolor=lightgray,linewidth=0.5pt](0,2.7,2.7)(2.7,2.7,2.7)
\pstThreeDLine[linecolor=lightgray,linewidth=0.5pt](0,0,2.7)(0,2.7,2.7)
\pstThreeDLine[linecolor=lightgray,linewidth=0.5pt](2.7,0,2.7)(2.7,0,0)
\pstThreeDLine[linecolor=lightgray,linewidth=0.5pt](2.7,2.7,0)(2.7,0,0)
\pstThreeDLine[linecolor=lightgray,linewidth=0.5pt](2.7,2.7,0)(0,2.7,0)
\pstThreeDLine[linecolor=lightgray,linewidth=0.5pt](2.7,2.7,0)(2.7,2.7,2.7)
}

\end{pspicture}

\begin{pspicture}(4,-0.3)
\psset{Alpha=50,Beta=20}

\rput[c](0.60,2.45){$\bullet$}
\rput[c](1.60,2.58){$\bullet$}
\rput[c](2.30,2.41){$\bullet$}
\rput[c](3.16,1.55){$\bullet$}
\rput[c](1.82,1.30){$\bullet$}
\rput[c](0.80,1.64){$\bullet$}

\rput[c](0.60,1.45){$\circ$}
\rput[c](1.10,1.18){$\circ$}
\rput[c](2.30,0.41){$\circ$}
\rput[c](3.05,0.25){$\circ$}
\rput[c](2.82,0.35){$\circ$}
\rput[c](1.82,0.40){$\circ$}
\rput[c](1.82,0.10){$\circ$}
\rput[c](1.72,-0.10){$\circ$}
\rput[c](0.90,0.74){$\circ$}
\rput[c](0.80,0.64){$\circ$}

\rput[c](1.6,3.9){Pareto frontier}
\pscurve[arrowscale=2,linecolor=gray]{->}(2.11,1.51)(4,1.4)(1.6,3.7)

\pspolygon[linestyle=none, linewidth=0pt,fillstyle=solid,fillcolor=black!50,opacity=0.4](0.60,2.45)(1.60,2.58)(2.30,2.41)(3.16,1.55)(1.82,1.30)(0.80,1.64)

\pstThreeDPut(-2.7,0,0){
\pstThreeDLine[linecolor=lightgray,linewidth=0.5pt](0,0,2.7)(2.7,0,2.7)
\pstThreeDLine[linecolor=lightgray,linewidth=0.5pt](0,0,0)(0,0,2.7)
\pstThreeDLine[linecolor=lightgray,linewidth=0.5pt](0,0,0)(2.7,0,0)
\pstThreeDLine[linecolor=lightgray,linewidth=0.5pt](0,0,0)(2.7,0,0)
\pstThreeDLine[linecolor=lightgray,linewidth=0.5pt](0,0,0)(0,2.7,0)
\pstThreeDLine[linecolor=lightgray,linewidth=0.5pt](0,2.7,2.7)(0,2.7,0)
\pstThreeDLine[linecolor=lightgray,linewidth=0.5pt](2.7,0,2.7)(2.7,2.7,2.7)
\pstThreeDLine[linecolor=lightgray,linewidth=0.5pt](0,2.7,2.7)(2.7,2.7,2.7)
\pstThreeDLine[linecolor=lightgray,linewidth=0.5pt](0,0,2.7)(0,2.7,2.7)
\pstThreeDLine[linecolor=lightgray,linewidth=0.5pt](2.7,0,2.7)(2.7,0,0)
\pstThreeDLine[linecolor=lightgray,linewidth=0.5pt](2.7,2.7,0)(2.7,0,0)
\pstThreeDLine[linecolor=lightgray,linewidth=0.5pt](2.7,2.7,0)(0,2.7,0)
\pstThreeDLine[linecolor=lightgray,linewidth=0.5pt](2.7,2.7,0)(2.7,2.7,2.7)
}

\end{pspicture}

\begin{pspicture}(-6,-0.6)(0,-0.6)
\psset{Alpha=50,Beta=20}

\rput[c](0.60,2.45){$\bullet$}
\rput[c](1.60,2.58){$\bullet$}
\rput[c](2.30,2.41){$\bullet$}
\rput[c](3.16,1.55){$\bullet$}
\rput[c](1.82,1.30){$\bullet$}
\rput[c](0.80,1.64){$\bullet$}

\rput[c](0.60,1.45){$\bullet$}
\rput[c](1.10,1.18){$\bullet$}
\rput[c](2.30,0.41){$\circ$}
\rput[c](3.05,0.25){$\circ$}
\rput[c](2.82,0.35){$\circ$}
\rput[c](1.82,0.40){$\circ$}
\rput[c](1.82,0.10){$\circ$}
\rput[c](1.72,-0.10){$\circ$}
\rput[c](0.90,0.74){$\bullet$}
\rput[c](0.80,0.64){$\bullet$}

\rput[c](1.6,3.9){Kaldor-Hicks region}
\pscurve[arrowscale=2,linecolor=gray]{->}(1.30,0.80)(4,1.4)(1.6,3.7)

\pspolygon[linestyle=none, linewidth=0pt,fillstyle=solid,fillcolor=black!50,opacity=0.4](0.60,2.45)(1.60,2.58)(2.30,2.41)(3.16,1.55)(1.82,1.30)(0.80,1.64)
\pspolygon[linestyle=none, linewidth=0pt,fillstyle=solid,fillcolor=black!50,opacity=0.4](0.60,2.45)(1.60,2.58)(2.30,2.41)(3.16,1.55)(0,0)%(0,1)

\pstThreeDPut(-2.7,0,0){
\pstThreeDLine[linecolor=lightgray,linewidth=0.5pt](0,0,2.7)(2.7,0,2.7)
\pstThreeDLine[linecolor=lightgray,linewidth=0.5pt](0,0,0)(0,0,2.7)
\pstThreeDLine[linecolor=lightgray,linewidth=0.5pt](0,0,0)(2.7,0,0)
\pstThreeDLine[linecolor=lightgray,linewidth=0.5pt](0,0,0)(2.7,0,0)
\pstThreeDLine[linecolor=lightgray,linewidth=0.5pt](0,0,0)(0,2.7,0)
\pstThreeDLine[linecolor=lightgray,linewidth=0.5pt](0,2.7,2.7)(0,2.7,0)
\pstThreeDLine[linecolor=lightgray,linewidth=0.5pt](2.7,0,2.7)(2.7,2.7,2.7)
\pstThreeDLine[linecolor=lightgray,linewidth=0.5pt](0,2.7,2.7)(2.7,2.7,2.7)
\pstThreeDLine[linecolor=lightgray,linewidth=0.5pt](0,0,2.7)(0,2.7,2.7)
\pstThreeDLine[linecolor=lightgray,linewidth=0.5pt](2.7,0,2.7)(2.7,0,0)
\pstThreeDLine[linecolor=lightgray,linewidth=0.5pt](2.7,2.7,0)(2.7,0,0)
\pstThreeDLine[linecolor=lightgray,linewidth=0.5pt](2.7,2.7,0)(0,2.7,0)
\pstThreeDLine[linecolor=lightgray,linewidth=0.5pt](2.7,2.7,0)(2.7,2.7,2.7)
}
\end{pspicture}

%\begin{pspicture}(-6,-0.6)(0,-0.6)
%    \psaxes[labels=none,ticks=none,linewidth=0.5pt,arrowscale=1.5]{->}(1.5,0.3)(1.5,0.3)(6.4,4.5)
%    %\rput{90}(1.2,2.4){Memorability}
%    %\rput[c](4.2,0){Adaptiveness}
%
%    \rput[c](2,4){$\bullet$}
%    %\psline{-}(2,4)(3.3,3.5)
%    \rput[c](3.3,3.5){$\bullet$}
%    %\psline{-}(3.3,3.5)(4.2,3)
%    \rput[c](4.2,3){$\bullet$}
%    %\psline{-}(4.2,3)(5.6,1.7)
%    \rput[c](5.6,1.7){$\bullet$}
%    %\psline{-}(5.6,1.7)(6.1,0.5)
%    \rput[c](6.1,0.5){$\bullet$}
%
%    %\psline{-}(1.9,3.5)(6.1,0.5)
%
%\pscurve[arrowscale=2,linecolor=gray]{->}(4.1,2.2)(6.2,2.2)(4.6,3.4)
%    \rput{-21}(3.2,3.92){Kaldor-Hicks region}
%    %\rput{-37}(4.2,2.2){Kaldor-Hicks region}
%
%\pspolygon[linestyle=none, linewidth=0pt,fillstyle=solid,fillcolor=black!50, opacity=0.4](2,4)(3.3,3.5)(4.2,3)(5.6,1.7)(6.1,0.5)(1.9,3.5)
%
%    \rput[c](2.5,3.5){$\bullet$}
%    \rput[c](2.8,3.2){$\bullet$}
%    \rput[c](3.8,2.9){$\bullet$}
%    \rput[c](4.1,2.6){$\bullet$}
%    \rput[c](5.0,1.7){$\bullet$}
%    \rput[c](5.4,1.2){$\bullet$}
%    \rput[c](2.2,1.7){$\circ$}
%    \rput[c](1.9,3.1){$\circ$}
%    \rput[c](2.3,2.5){$\circ$}
%    \rput[c](2.8,2.5){$\circ$}
%    \rput[c](4.2,1.1){$\circ$}
%    \rput[c](3.0,0.6){$\circ$}
%    \rput[c](2.8,2.5){$\circ$}
%    \rput[c](4.1,0.7){$\circ$}
%    \rput[c](5.1,0.8){$\circ$}
%    \rput[c](3.1,1.7){$\circ$}
%
%\end{pspicture}

\caption{Illustrative example. (Left) The dominance operator: neither $a$ or $b$ dominates each other, but $b$ dominates $c$. (Middle) Points lying in the Pareto frontier. (Right) Points inside the Kaldor-Hicks region.}
\label{fig:ex2}
\end{figure*}


\paragraph*{\bf{Definition 4}} Training set $\mathcal{P}_n=\{d_1, d_2, \ldots, d_m\}$ is said to be Pareto-efficient at time step $n$, if $\mathcal{P}_n\subseteq\mathcal{D}_n$ and there is no pair of messages $(d_i, d_j)\in\mathcal{P}_n$ for which $d_i$ dominates $d_j$.\\

Messages that are not dominated
by any other message, lie on the Pareto frontier~\cite{palda@book}. Therefore, by definition, the Pareto-efficient training set at time step $n$, $\mathcal{P}_n$, is composed by all the messages lying in the Pareto frontier that is built from $\mathcal{D}_n$.
There are efficient algorithms for building and maintaining the Pareto frontier, and we employed the algorithm proposed in~\cite{operator} which ensures $O$($|\mathcal{D}_n|$) complexity.
%Finally, dominated messages are discarded from $\mathcal{D}_n$, since for each dominated message there is at least one message in $\mathcal{P}_n$ that beats it in at least one utility measure.
We denote the process of exploiting Pareto-efficient training sets as
Pareto-Efficient Selective Sampling, or simply PESS. Figure~\ref{fig:ex2} (Middle) shows an illustrative example of a Pareto frontier built from arbitrary points in the utility space.
%These messages are likely to offer a good balance between adaptiveness and memorability.

\subsection*{Kaldor-Hicks Region}
%\paragraph*{\bf{Kaldor-Hicks Region}}
The PESS strategy follows a stringent criterion, which tends to select only few messages to compose the training sets. As a result, the training sets may become excessively small and prone to noise.
The Kaldor-Hicks criterion, on the other hand,
follows a cost-benefit analysis and circumvents the
small training set problem by stating that efficiency is achieved if
those that are made better off could in theory compensate those that are made worse off. Thus, under the Kaldor-Hicks criterion, an utility measure can compensate other utility measures, and therefore, this criterion selects messages that are located inside a region which is below the Pareto frontier. To define this region we must first define the overall utility of a message.

\paragraph*{\bf{Definition 5}} Assuming that all measures are equally important, the overall utility of an arbitrary message $d_i\in\mathcal{D}_n$ is:

\begin{equation}
\label{eq:cost}
%\nonumber
U(d_i)=U_s(d_i)+U_t(d_i)+U_r(d_i)
\end{equation}

\noindent That is, the overall utility of a message is given as the sum of its utility measures. Also, the baseline message, which is denoted as $d^*$, is defined as:

\begin{equation}
%\nonumber
d^*=\{d_i\in\mathcal{P}_n | \forall d_j\in\mathcal{P}_n: U(d_i)\leq U(d_j)\}
\end{equation}

\noindent That is, the baseline is the message lying in the frontier for which the overall utility assumes its lowest value.\\

The Kaldor-Hicks region is composed of messages for which the overall utility is not smaller than the baseline overall utility. Such baseline utility is the utility associated with the message lying in the Pareto frontier for which the overall utility is the lowest.

%\begin{algorithm}[t!]
%  \caption{\textbf{Pareto-Efficient Selective Sampling}}
%  \label{alg:pareto}
%  \begin{algorithmic}[1]
%    \Require Messages in $\mathcal{D}_n$
%    \Statex
%    \State $\mathcal{P}_n \gets \emptyset$
%    \ForAll {messages $d \in \mathcal{D}_n$}
%        \State $notDominated \gets True$
%        \ForAll {messages $p \in \mathcal{P}_n$}
%            \If{$d$ dominates $p$}
%                  \State Remove $p$ from $\mathcal{P}_n$
%                \ElsIf{$p$ dominates $d$}
%              \State insert $p$ into $\mathcal{P}_n$
%                  \State $notDominated \gets False$
%                  \State $break$
%                \EndIf
%        \EndFor
%        \If{$notDominated$}
%                \State insert $d$ into $\mathcal{P}_n$
%        \EndIf
%    \EndFor
%    \State $\mathcal{D}_n \gets \mathcal{P}_n$
%    \State \textbf{return} $\mathcal{P}_n$
%  \end{algorithmic}
%\end{algorithm}

\paragraph*{\bf{Definition 6}} Training set $\mathcal{K}_n=\{d_1, d_2, \ldots, d_m\}$ is said to be Kaldor-Hicks-efficient at time step $n$, if $\mathcal{P}_n\subseteq\mathcal{K}_n\subseteq\mathcal{D}_n$, and there is no message $d_i\in\mathcal{K}_n$ such that $U(d^*)>U(d_i)$.\\

We denote the process of exploiting Kaldor-Hicks-efficient training sets as
Kaldor-Hicks-Efficient Selective Sampling, or simply KHSS. Figure~\ref{fig:ex2} (Right) shows an illustrative example of a Kaldor-Hicks region built from arbitrary points in the utility space.

%\begin{algorithm}[t!]
%  \caption{\textbf{Kaldor-Hicks-Efficient Selective Sampling}}
%  \label{alg:kh}
%  \begin{algorithmic}[1]
%    \Require Messages in $\mathcal{D}_n$ and messages in $\mathcal{P}_n$
%    \Statex
%
%    \State $\mathcal{K}_n \gets \emptyset$
%    %\State $min \gets \infty$
%    %\ForAll{messages $p \in \mathcal{P}_n$}
%    %    \State $k \gets U_s(p)+U_t(p)+U_r(p)$
%    %    \If{$k < min$}
%    %        \State $min \gets k$
%    %    \EndIf
%    %\EndFor
%    \State $d^* \gets \{d_i\in\mathcal{P}_n | \forall d_j\in\mathcal{P}_n: U(d_i)\leq U(d_j)\}$
%
%    \ForAll{messages $d \in \{\mathcal{D}_n-\mathcal{P}_n\}$}
%        %\State $q \gets U_s(d)+U_t(d)+U_r(d)$
%        %\If{$q \geq min$}
%        \If{$U(d) \geq U(d^*)$}
%            \State insert $d$ into $\mathcal{K}_n$
%        \EndIf
%    \EndFor
%    \State $\mathcal{D}_n \gets \mathcal{K}_n$
%    \State \textbf{return} $\mathcal{K}_n$
%  \end{algorithmic}
%\end{algorithm}

\input{case}
\section{Conclusions}

This paper focused on sentiment analysis on Twitter streams. We have introduced new algorithms for active training-set formation, which we denote as Pareto-Efficient Selective Sampling (PESS) and Kaldor-Hicks Selective Sample (KHSS). The proposed algorithms provide the resulting classifier with memorability and adaptiveness. We formalized the selective sampling process as a multi-objective optimization procedure, which finds a proper balance between adaptiveness and memorability.
Adaptiveness is assessed by computing the distance in time and space between the target message and the candidate ones. Also, candidate messages are randomly shuffled, thus providing memorability to the resulting classifier. The message utility space is composed by such dimensions, and we compute the Pareto Frontier in this space in order to pick up messages satisfying the Pareto improvement condition, finding a proper balance between adaptiveness and memorability. The Kaldor-Hicks criterion enables memorability to compensate adaptiveness, or vice-versa. A systematic evaluation involving recent events demonstrated the effectiveness of our algorithms.

As future work, we intend to extend our strategies for algorithms that do not depend on manual labeling.


\balance

\section{Acknowledgements}

This research was partially supported by CNPq, Capes, Finep, Fapemig, and by project INCTWeb (CNPq grant number 573871/2008-6).


\bibliographystyle{abbrv}
\bibliography{adriano,references}

\end{document}
